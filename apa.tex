\documentclass[a4paper,man,natbib,donotrepeattitle]{apa6}

\usepackage[ngerman]{babel}
\usepackage[utf8]{inputenc}
\usepackage{amsmath}
\usepackage{graphicx}
\usepackage{here} 
\usepackage{float}
\restylefloat{figure}
\usepackage[colorinlistoftodos]{todonotes}
\usepackage[hyphens]{url}
\usepackage{setspace}
\usepackage{textcomp}
\usepackage{microtype}
\usepackage{ragged2e}
\setstretch{1.5}

%-----------------------Tabellen-----------------------------------
\usepackage{tabularx}                          %Tabellenspalten zusammenlegen
\usepackage{multirow}                          %Tabellezeilen zusammenlegen
\usepackage{booktabs}                                   %schöne Tabellen
\usepackage{longtable}                        %mehrseitige Tabellen
\usepackage{rotating}                           %vertikale Tabellenspaltenausrichtung
\usepackage{array}                                          %andere Tabellenspaltenausrichtung
\usepackage{paralist}
\usepackage{colortbl} % Paket für die Farbe
\usepackage{xcolor} % Paket für farbige Texte
\usepackage{epigraph}

\title{Mitarbeitermotivation fördern / Personelle Anreizsysteme}
\shorttitle{Mitarbeitermotivation fördern}
\author{Carsten Meiser und Andreas Schillinger}
\affiliation{Hochschule Karlsruhe für Technik und Wirtschaft}

\abstract{Das ist das Abstrakt}


\hyphenation{Geschäfts-modell Marken-bekanntheit Werbe-botschaft}
\setcounter{secnumdepth}{3}
\setcounter{tocdepth}{3}

\begin{document}
\maketitle
\newpage
\tableofcontents
\newpage

\justifying
\section{Einleitung}

Gegenwärtig finden sich in den deutschen Medien vermehrt Beiträge zum Thema Mitarbeitermotivation mit Bezug auf die Herausforderungen in Unternehmen und der Zukunft, bedingt durch den Fachkräftemangel und eine älter werdende Gesellschaft. \citep{Zeit.03.04.2014}

Die Volksweisheit \glqq Geld allein macht nicht glücklich\grqq kennt wahrscheinlich jeder und trotzdem finden sich in den Unternehmen überwiegend Anreizsysteme auf monetärer Basis. Führungskräfte machen im Sinne des Unternehmens meistens alles richtig und doch machen sie nicht das Richtige. Der Faktor Mensch bleibt oft auf der Strecke. \citep[S. 16]{Seelbach.2011}

Viele Unternehmen haben erkannt, dass motivierte Mitarbeiter ein nicht zu unterschätzender Erfolgsfaktor sind. Der Einsatz von monetären Anreizsystemen ist weit verbreitet und eine Maßnahme, die Unternehmen gerne einsetzen um steuernd auf die Motivation ihrer Mitarbeiter einzuwirken. Oft in Vertretung eines variablen Gehaltsanteils, wird dem Mitarbeiter eine hohe Leistung extra vergütet. Genau dieser variable Gehaltsanteil steht bei vielen Mitarbeitern jedoch nicht für  eine Extramotivation, sondern für einen hohen Leistungsdruck. Oft ist das Fixgehalt so niedrig, dass der variable Gehaltsanteil als fester Bestandteil der Entlohnung einkalkuliert werden muss. Damit stellt sich die Frage, welche Instrumente Unternehmen noch zur Motivation ihrer Mitarbeiter außerhalb, bzw. in Kombination zu monetären Anreizsystemen, zur Verfügung stehen. \citep{Nowka.2013}

Ein besonderes Augenmerk wird auf die Erkenntnisse der Neurowissenschaften des  20.Jahrhunderts bis heute gelegt. Hierzu trugen unter anderem die Entwicklung von bildgebenden Messmethoden bei, die es ermöglichen die Aktivitäten im Gehirn direkt darzustellen. Dadurch ist es möglich, nicht nur das äußerliche Verhalten von Probanden zu beschreiben, sondern auch die im Gehirn ablaufenden Prozesse. 

Das Ziel dieser Arbeit ist die Entwicklung eines Konzepts für Unternehmen, die Maßnahmen zur Förderung der Mitarbeitermotivation ergreifen wollen um nachhaltig erfolgreich zu sein. Als Grundlage dienen dabei die klassischen Motivationstheorien und die Erkenntnisse der neurowissenschaftlichen Forschung.

\newpage
\section{Fachliche Grundlagen} % (fold)
\label{sub:einordnung_und_definition}

\subsection{Begriffserklärung} % (fold)
\label{ssub:E-Entrepreneurship}
Die Begriffe Anreiz, Motiv und Motivation werden im Alltag oft synonym verwendet. Zur Förderung des Verständnisses dieser Arbeit, werden im Folgenden diese Begriffe bestimmt und gegeneinander abgegrenzt. [Mareen Nowka, 2013, S.19]

\subsubsection{Motiv} % (fold)
\label{ssub:E_Entrepreneurial_branding}
Ein Motiv ist die Erwartungshaltung, die einer Handlung zugrunde liegt. Somit kann das gesamte menschliche Verhalten als Motiv-gesteuert angesehen werden. Die Definition und Bewertung der jeweiligen Motive erfolgt individuell und subjektiv. Zusätzlich können Motive durch Umweltreize sowohl verstärkt wie vermindert werden. [Mareen Nowka, 2013, S.19f]

Desweiteren kann zwischen primären und sekundären Motiven unterschieden werden. Erstere werden durch physiologische Vorgänge im Organismus hervorgerufen, z.B. Hunger und Durst. Sekundäre Motive hingegen sind von der Umwelt erlernt, z.B. das Verdienen von Geld. [Mareen Nowka, 2013, S.20, zitiert nach Jung, 2011]


\paragraph{Leistungsmotivation} % (fold)
\label{ssub:der_Markenbegriff}
Das Motiv der Leistung das bis heute am meisten untersuchte Motiv. Es ist das Streben nach Exzellenz, die Suche nach der Herausforderung, die eigenen Fähigkeiten unter Beweis zu stellen. Vorraussetzung dafür ist immer eine Vergleichbarkeit. Entweder aus eigener Erfahrung, mit der Leistung Anderer oder allgemein anerkannter Gütestandards. Der Antrieb zur Leistung geht dabei von der Person selbst aus. Die zugrunde liegende Definition, was Leistung ist, kann sich dabei je nach kultureller und sozialer Zugehörigkeit unterscheiden. [J. C. Brunstein, H. Heckhausen, 2010, S.145f]

\paragraph{Soziale Bindung: Anschlussmotivation und Intimitätsmotivation} % (fold)
\label{ssub:markenaufbau_markenfuehrung}
Den großteil seines Lebens verbringt der Mensch in Gesellschaft Anderer. Die Interaktion reicht dabei von beziehungslosem Miteinander über anonyme Konkurrenz bis hin zu aggressivem Gegeneinander. Die Voraussetzung zur zwischenmenschlichen Kommunikation ist dem Menschen dabei wie allen Säugetieren angeboren. Maßgeblichen Einfluss auf die Art der Kommunikation haben Emotionen. Sie wirken als Signale des Emotionsausdrucks und des Motivationszustandes einer Person. Ihr subjektives Erleben bildet in der Summe den aktuellen Motivationszustand einer Person. Zur Äußerung kommt dies in vielfältiger Form verbaler und nonverbaler Kommunikationsformen wie Gestik, Mimik, Stimmführung, Körperhaltung und Distanzveränderungen zu seinem Gegenüber.
Das Intimitätsmotiv entspricht umgangssprachlich der \glqq Liebe\grqq. Das Ziel ist die Schaffung und Erhaltung einer eng vertrauten Beziehung, sich gegenseitig austauschender Zweisamkeit.
Es gibt aber auch ein große Menge sozialer Interaktion, die nicht dem Anschlussmotiv angerechnet werden kann. Dazu zählen zum Beispiel das Messen mit Anderen, das Beherrschen anderer Menschen oder auch das umgangssprachliche “prahlen” vor Anderen. [Sokolowski, Heckhausen, 2010, S.193f]

\paragraph{Machtmotivation} % (fold)
Schon immer wurde versucht, die Macht in ihrer vielfältigen Form und ungleichen Verteilung zu Erklären, Rechtzufertigen oder Herauszufordern. Selbst bei nichtmeschlichen Primaten spielen Macht und Dominanz eine zentrale Rolle. Jedem Sozialgebilde jeglicher Form liegt eine differenzierende Verteilung der Macht zugrunde. Es scheint, dass es keines gibt, welches ohne diese dauerhaft überlebensfähig wäre. [Schmalt, Heckhausen, 2010, S.211]

Der motivationspsychologische Ansatz beschreibt Macht als einen Austauschprozess zwischen verschiedenen Individuen. Die Individualebene betrachtet dabei die Motive, der Austauschprozess die Verhaltens- und Situativkomponente. [Schmalt, Heckhausen, 2010, S.211f]

\subsubsection{Motivation}
Gabler beschreibt die Motivation als den \glqq Zustand einer Person, der sie dazu veranlasst, eine bestimmte Handlungsalternative auszuwählen, um ein bestimmtes Ergebnis zu erreichen und der dafür sorgt, dass diese Person ihr Verhalten hinsichtlich Richtung und Intensität beibehält\grqq.

Drumm versteht \glqq unter Motivation (...) ein[en] geistig-seelische[n] Antrieb zur Steuerung und zum Vollzug des Handelns und Verhaltens \grqq [Drumm, 2008, S.384].

Bei den genannten Defintionen lässt sich entnehmen, dass die Motivation maßgeblich eine Handlung vorantreibt. 

Nach Haubruck wird unter Motivation \glqq im Unternehmenskontext (...) in der Regel die Erhöhung und Förderung, teilweise auch die Aufrechterhaltung der Leistungsbereitschaft verstanden.\grqq [Haubruck, 2004, S.109]

\paragraph{Intrinsische Motivation}
Befriedigt die Tätigkeit als solche jemanden, empfindet er Spaß und Freude bei der Ausführung dieser, wird von intrinsischer Motivation gesprochen. Es ist der innere Drang etwas zu tun. [Nowka, 2013, S.40f] 

Der Flow-Effekt ist ein Zustand, bei welchem \glqq […] es sich um das selbstreflexionsfreie, gänzliche Aufgehen in einer glatt laufenden Tätigkeit [handelt], bei der man trotz voller Kapazitätsauslastung das Gefühl hat, den Geschehensablauf noch gut unter Kontrolle zu haben\grqq [Rheinberg, 2010, S.380]
Das selbstreflexionsfreie Arbeiten kommt dabei u.A. durch Vergessen der Zeit und Ausblendung aller persönlichen Sorgen zum Vorschein. 

\paragraph{Extrinsische Motivation}
Im Gegensatz zur intrinsischen Motivation steht bei der Extrinsischen nicht die Handlung an sich im Vordergrund, sondern die mögliche Belohung bzw. nicht-Bestrafung. Das Handeln basiert auf sekundären Motiven. So bedarf es eines, wie auch immer gearteten, von außen gesetzten Anreizes, zur Durchführung der Handlung bzw. Tätigkeit. [Nowka, 2013, S.41f]

\subsubsection{Anreiz}
Ein Anreizsystem besteht aus mehreren Anreize im Wirkungsverbund mit dem Ziel, erwünschte Verhaltensweisen herbeizuführen und unerwünschte Verhaltensweisen zu unterdrücken. Aufgabe eines Anreizsystems ist es, anstatt mit explizit ausformulierten Verhaltensanweisungen, mit Zielvorgaben das gewünschte Verhalten implizit zu fördern bzw. unerwünschtes zu unterbinden. Zur Vereinfachung greifen Anreizsysteme meist nur wenige als typisch empfundene Bedürfnisse auf. Voraussetzung dafür sind zumindest begrenzte Entscheidungs- und Verhaltensfreiheiten der darin eingebundenen Individuen [Drumm, 2008, S.458].

\subsection{Motivationstheorien}
Die Motivationstheorien versuchen, den Zusammenhang von Motivation und menschlichem Verhalten zu beschreiben. Aus der Erkenntnis, dass die Motivation eines Mitarbeiters maßgeblich seine Leistung beeinflusst, haben sich drei wesentliche Theorien entwickelt:

- Inhaltstheorien beschäftigen sich mit \glqq der Art, Inhalt und Wirkung der Bedürfnisse von Individuen\grqq [Drumm, 2008, S.391].

- Prozesstheorien betrachten den Prozess der Motivation losgelöst von Bedürfnisinhalten [Drumm, 2008, S.391].

\subsubsection{Inhaltstheorien}
\paragraph{Die Zwei-Faktoren-Theorie von Frederick Herzberg}
Nach Herzberg unterteilen sich die Grundbedürfnisse eines Menschen in Hygienebedürfnisse und  Motivationsbedürfnisse.

Erstere führen bei Nichterfüllung zu Unzufriedenheit und werden daher als Selbstverständlichkeit angesehen. Sie führen bei Erfüllung jedoch nicht zu Zufriedenheit. Als Beispiele seien hier die Sicherheit des Arbeitsplatzes sowie eine angemessene Bezahlung genannt. [Nowka, 2013, S.26]

Die Erfüllung von Motivationsbedürfnissen hingegen führt zu Zufriedenheit. Herzberg bezeichnet diese als direkt auf die Tätigkeit bezogene Faktoren wie Anerkennung und Verantwortung. Er zielt damit auf die intrinsischen Arbeitsbedürfnisse eines Mitarbeiters ab. [Nowka, 2013, S.26f]

Kritik an Herzbergs Theorien bezieht sich vor allem auf die situative Abhängkeit der Motivatorien und Frustratoren sowie fehlende soziokulturelle Einflüsse. [Drumm, 2008, S.396]

\paragraph{Die Motivationstheorie von David C. McClelland}
McClelland setzt das Machtmotiv in das Zentrum seiner Theorie. Danach muss ein Vorgesetzter seinen Mitarbeitern das Gefühl eigener Macht vermitteln, ohne jedoch den Anschein der Unterwerfung zu erwecken. Grundvoraussetzung dafür ist, dass Autonomie und Verantwortung für das eigene Handeln bei jedem Mitarbeiter als äußere Bedingung gegeben sind. 
Zur Ausübung des Machtmotivs stellt McClelland drei lernbare Formen dar. Sie unterscheiden sich im Maß der Eigenverantwortung, welche den Mitarbeitern übertragen wird. Die einfachste Form ist die Persönliche Herrschaft der Führungskraft, gefolgt von einer uneigennützigen Machtausübung bis hin zur selbstlosen Führerschaft [Drumm, 2008, S.396ff].

Kritisiert an McClellands Theorie wird vor allem der singuläre Fokus auf das Machtmotiv sowie das fehlende Zeitverhalten. Zugute zuhalten ist McClelland hingegen die zumindest teilweise Legitimation von Konzepten partizipativer Führung [Drumm, 2008, S.398]

\subsubsection{Prozesstheorien}
\paragraph{Die VIE-Theorie von Vroom}
Die VIE-Theorie basiert auf den drei Variablen Valenz, Instrumentalität und Erwartung. Eine ihrer Kernaussagen ist es, dass Mitarbeiter zur Arbeit motiviert sind, wenn sie darin einen Weg sehen, ihre persönlichen Ziele zu erreichen. [Nowka, 2013, S.30]
 
Vroom konzentriert sich dabei auf den Prozessablauf. Die Valenz steht dabei für die \glqq […] persönlich empfundene Bedeutung des Handlungsergebnisses\grqq [Nowka, 2013, S.31]. Die Auslegung ist Situationsabhängig und kann sich mit der Zeit ändern. Die Instrumentalität entspricht der Tauglichkeit einer Handlung zur Zielerreichung. Die Erwartung lässt sich in Handlungs-Ergebnis-Erwartung und Ergebnis-Folge-Erwartung unterteilen.
Erstere beschreibt die subjektive Wahrscheinlichkeit der Zielerreichung einer Handlung unter Berücksichtigung weiterer Faktoren wie der individuellen Qualifikation und dem Grad der dafür notwendigen Anstrengung.
Letztere steht für die subjektive Wahrscheinlichkeit, die erwartete Belohnung tatsächlich zu erhalten [Nowka, 2013, S.31].
 
Die Motivation entspricht dem Produkt aller drei Variablen. Das bedeutet aber auch, wenn eine der Variablen null ist, so ist die Motivation gleich null [Nowka, 2013, S.31].
 
Kritik an der VIE-Theorie gibt es an der Prämisse, dass allgemein die Nutzenmaximierung als Ziel jedes Individuums unterstellt wird [Drumm, 2008, S.401].

\subsection{Anreizsysteme}
Ein Anreizsystem besteht aus mehreren Anreizen im Wirkungsverbund mit dem Ziel, erwünschte Verhaltensweisen herbeizuführen und unerwünschte Verhaltensweisen zu unterdrücken. Aufgabe eines Anreizsystems ist es, anstatt mit explizit ausformulierten Verhaltensanweisungen, mit Zielvorgaben das gewünschte Verhalten implizit zu fördern bzw. unerwünschtes zu unterbinden. Zur Vereinfachung greifen Anreizsysteme meist nur wenige als typisch empfundene Bedürfnisse auf. Voraussetzung dafür sind zumindest begrenzte Entscheidungs- und Verhaltensfreiheiten der darin eingebundenen Individuen. Weitergehende Definitionen eines Anreizsystems im Unternehmenskontext umfassen auch Parameter wie Arbeitsbedingungen, Unternehmensimage sowie das Führungssystem des Unternehmens [Drumm, 2008, S.458].

\subsection{Neurobiologische Grundlagen}
Im Verlauf dieses Abschnitts werden die für diese Arbeit relevante Teile des Gehirns und deren Zuständigkeiten vorgestellt. Aufgrund der sehr komplexen Zusammenhänge wird hier nur Sachverhalte sehr allgemein beschrieben. 

\subsubsection{Das Gehirn}
Das Gehirn und das Rückenmark bilden zusammen das Zentralnervensystem. Beide Teile sind \glqq sowohl funktionell als auch anatomisch untrennbar\grqq miteinander verbunden. [Kirschbaum 2008, S.105] Im Durchschnitt wiegt das menschliche Gehirn 1500g und umfasst nach Schätzungen  ca. 100 Milliarden Nervenzellen die durch ca. 100 Billionen Synapsen miteinander verbunden sind.

Das Gehirn besteht im Wesentlichen aus fünf Teilen, dem Groß- oder Endhirn, dem Hinterhirn (Kleinhirn), dem Zwischenhirn, dem Mittelhirn und dem Nachhirn (verlängertes Rückenmark).

Hinterhirn (Kleinhirn und Pons)
Das Kleinhirn, auch Zerebellum, ist die zweitgrößte Gehirnstruktur und unter anderem zuständig für die \glqq Koordination, Feinabstimmung, Planung sowie das Erlernen von Bewegungsabläufen\grqq. [Kirschbaum, S. 301]

Zwischenhirn
Das Zwischenhirn beherbergt die funktionellen Struktur des Thalamus und des Hypothalamus. Der Hypothalamus steuert \glqq lebenswichtige Funktionen wie Hunger, Durst, Fortpflanzungsverhalten und Körpertemperatur.\grqq [Kirschbaum, S. 305]

Hirnstamm
Der Hirnstamm beherbergt die \glqq Zentren zur Kontrolle der Atmung, des Kreislaufs sowie verschiedener Reflexe (u.a. Brechreflex, …).\grqq [Kirschbaum, S. 127] Das Versagen eines Teilbereichs (Medulla oblongata) hat den direkten Tod zur Folge. [Nowka, S. 6]

Mittelhirn
Das Mittelhirn steuert u.a. die Bewegung der Augen.

\subsubsection{Großhirn}
Das Großhirn ist der größte Teil des Gehirns und ist in zwei miteinanderverbundene Hälften aufgeteilt. Die linke Seite steuert die rechte Körperhälfte und die rechte Seite steuert die linke Körperhälfte. Es ist unter anderem zuständig für die \glqq Steuerung komplexer Prozesse wie Wahrnehmung, Lernen, Motivation und Denken.\grqq [Kirschbaum, S.115 ]

Großhirnrinde
Die Großhirnrinde (Cortex) ist der äußerlich erkennbare Teil des Großhirns und vergrößert durch ihre Faltung nicht nur die Oberfläche des Gehirns, sondern optimiert auch die Informationsverarbeitung und -wege. [Nowka, S. 8]

Frontallappen (Stirnlappen)
Der Frontallappen, auch präfrontaler Cortex, ist unter anderem am bewussten Erleben von Freude beteiligt [Nowka, S.12]. 
Emotion 

Parietallappen (Scheitellappen)
“Körperwahrnehmung”

Occipitallappen (Hinterhauptlappen)
“Sinnesreize die über die Sehbahn”

Temporallappen (Schläfenlappen)
Sprache, akustische Reize; Gerüche

\subsubsection{Das limbische System}
Das limbische System ist die Zusammenfassung mehrerer Strukturen zu einem funktionelle System. [Weber S. 36 Artikel von Armin Derouiche] Nach [Kirschbaum, S.270 ] ist es das  \glqq Zentrum der Emotion, der Motivation und des Lernens im menschlichen Gehirn\grqq. \glqq Inzwischen ist bekannt, dass das limbische System an emotionalen und motivationalen Aspekten vielfältiger Verhaltensfunktionen (wie z. B. Lernen und Gedächtnis) beteiligt ist, was u. a. durch extensive Verbindungen limbischer Strukturen zum Neokortex und (anderen) Kernen des Hypothalamus ermöglicht wird.\grqq [Kirschbaum, S.270 ]

Hippocampus
Der Hippocampus ist neben der räumlichen Orientierung auch für das deklarative Gedächtnis zuständig. Es kann also zugeordnet werden, ob eine Situation neu oder bereits bekannt ist. [Weber S. 36 Artikel von Armin Derouiche]

Amygdala
\glqq Durch ihre Aktivierung werden Affekte ausgelöst (Wut, Angst, Ekel). Sie hat Verbindung zum Hypothalamus im Zwischenhirn, der Steuerungszentrale des vegetativen Nervensystems, so
dass u.a. erhöhter Herzschlag, Blutdruck und Schweißausbruch ausgelöst werden
können.\grqq [Weber S. 36-37 Artikel von Armin Derouiche]

Die Amygdala ist für das \glqq emotionale Färben von Informationen\grqq zuständig.  [Kirschbaum, S. 12] Das bedeutet, dass die Amygdala z.B. Situationen oder Orte durch \glqq emotionales Lernen\grqq mit einer Emotion wie Angst verknüpfen kann. Das führt z.B. dazu, dass durch dieses \glqq emotionale Gedächtnis\grqq ein Ort oder Situation mit einer Emotion wie Angst verknüpft werden kann. [Weber S. 37 Artikel von Armin Derouiche]   

Die Botenstoffe des Hirns

Angst: Stresshormon Cortisol

\glqq Dopamin erzeugt das Gefühl von Wohlbefinden und versetzt in einen Zustand von Konzentration und Handlungsbereitschaft. Dopamin steht für das Gefühl von: \glqq Ich will etwas tun!\grqq \grqq [Seelbach, S. 18]
\glqq Opioide wirken positiv auf das Ich-Gefühl, auf die emotionale Stimmung und die Lebensfreude. Das korrespondierende Gefühl heißt: \glqq Es macht Spaß, etwas zu tun!\grqq \grqq[Seelbach, S. 18]
\glqq Oxytocin ist eine Art Bindungsstoff. In Fachkreisen wird er auch \glqq Sozialkleber\grqq genannt. Er ist sowohl Ursache als auch Wirkung von Bindungserfahrungen. So konnte nachgewiesen werden, dass Menschen als Folge einer geschäftlichen Transaktion, in denen ihnen Vertrauen entgegengebracht wurde, erhöhte Oxytocin-Werte aufweisen. Oxytocin steht für: \glqq Ich setze mich für diejenigen ein, die mich mögen!\grqq \grqq[Seelbach, S. 18]


Gyrus cinguli

Nucleus accumbens
reagiert auf positive, lustvermittelnde Reize. Wird durch Dopamin aktiviert, steuert die Freisetzung von Endorphinen, was zu einem Glückgsgefühl führt. \glqq Eine verminderte Dopaminfreisetzung … kann zum Verlust jeglicher Motivation führen\grqq [Nowka, S. 11]. 
Nucleus accumbens gehört zum Belohnungssystem
Laut Graufik: gibt Dopamin ab [Nowka, S.12].

\subsubsection{Mesolimbisches Dopaminsystem}
Urpsrung im ventralen tegmentalen Areal (VTA) - Zentrum der Dopaminausschüttung:
\glqq Seine Aktivierung löst Empfindungen und Verhalten aus wie Wohlempfinden, erotisches
Empfinden, \glqq incentive behaviour\grqq, Appetenzverhalten und Sucht.\grqq [Weber S. 37 Artikel von Armin Derouiche] 

Ablauf: 
Reiz löst im VTA eine Dopaminausschüttung beim Nucleus Accumbens aus.
Menge des Dopamins ist ein Maß dafür, wie sehr etwas gewollt wird, nicht ob es wirklich gemocht wird. 
Dopamin dockt dann am präfrontalen Cortex an, wo das Bewusstsein liegt. Entscheidung wird gefällt (z.B. etwas zu essen). Nach Befriedigung (wenn sie gut war) meldet die Großhirnrinde ein positives Erlebnis ans VTA. Danach erfolgt eine Serotoninausschüttung -> wirkt beruhigend und befriedigend.

\subsection{Bildgebende Messmethoden}
Das Gebiet der Psychophysiologie umfasst die noninvasive Erforschung der Zusammenhänge zwischen psychologischen Prozessen und der physiologischen Aktivitäten. Zu den ersten Messverfahren rein elektrophysiologische Verfahren. In den letzten Jahren kommen vermehrt bildgebende Verfahren zum Einsatz. [Kirschbaum, S. 232]
Die Entwicklung dieser bildgebender Messmethoden lässt eine detaillierte Analyse der Wirkung von Anreizen und die ablaufenden Prozesse im Gehirn  zu. Mit Hilfe dieser dieser neuen Möglichkeiten konnten und können neue neurowissenschaftliche Erkenntnisse und Zusammenhänge erforscht werden.  [Nowka S.15]
 
Messungen mit Hilfe der Elektroenzephalographie (EEG) waren die ersten, welche es ermöglichten das Gehirn in vivo zu untersuchen. [Weber, S.43]  Die Magnetresonanz-Enzephalographie (MEG), die Positronen-Emissions-Tomographie (PET)  und die funktionelle Magnetresonanztomographie (fMRT) sind die modernen bildgebenden Verfahren, wobei die fMRT die zurzeit beste Methode ist, emotionale Reaktionen und kognitive Leistungen im Gehirn zu untersuchen [Weber, 2011, S. 50].
 
Mit Hilfe des aktuellen Forschungsstandes und den \glqq Messmethoden können Anreize anhand ihrer neurobiologischen Wirkung bewertet und verbessert werden.\grqq [Nowka S. 17]
\section{Aktueller Stand der Forschung}
Eine Person zu motivieren bedeutet Einfluss auf Ihre zukünftigen Entscheidungen zu nehmen. Jeder Entscheidung liegt mindestens ein Motiv zugrunde. Unklar ist dabei jedoch, wie hoch der Einfluss auf die Entscheidungsfindung ist. Früher wurden zur Untersuchung der Wirkung oft Versuchspersonen mit Anreizen motiviert und ihr Verhalten betrachtet. Der im Gehirn ablaufende Prozess zwischen Wahrnehmung des Anreizes und Entscheidungsfindung konnte nicht betrachtet werden. Heute ist dies mit den modernen Methoden der Neurowissenschaften möglich. \citep[S. 59]{Nowka.2013}
Die Neurowissenschaft bietet eine Möglichkeit, den Entscheidungsfindungsprozess genauer untersuchen zu können. Dabei gilt es zu berücksichtigen, dass jede Entscheidung von Informationen abhängt, die zu diesem Zeitpunkt im Gehirn eines Individuums gespeichert sind oder gerade verarbeitet werden. \citep[S. 60]{Nowka.2013}

\subsection{Neurowissenschaftliche Gehirnsysteme}
Die Wirkung eines Anreizes lässt sich heutzutage Mittels der Methoden der Neurowissenschaften (siehe \ref{sec:NeurobiologischeGrundlagen}) nachweisen, womit es möglichist, das Potenzial verschiedener Anreize aufzuzeigen. Die folgenden funktionalen Systeme sind im Gehirn vor allem für Führung und Motivation verantwortlich: 

\begin{APAitemize}
\item das Belohnungssystem
\item das Erinnerungssystem
\item das Emotionssystem
\item das Entscheidungssystem
\end{APAitemize}

\glqq Diese vier Systeme sind wesentlich dafür verantwortlich, ob und in welchem Umfang Führung und Motivation gelingen.\grqq \citep[S. 16]{Seelbach.2011}

\subsubsection{Belohnungssystem}
\label{sec:Belohnungssystem}
Das Belohnunggsystem umfasst unter anderem das, bereits erwähnte, ventrale tegmentale Areal (VTA) des Mittelhirns und den präfrontalen Cortex (siehe \ref{sec:dasGehirn}) und ist nicht dazu vorgesehen, dauerhaftes Wohlbefinden zu erzeugen. Vielmehr geht es bei den Prozessen im Belohnungszentrum darum, durch einen Wechsel zwischen Aktivierung und Deaktivierung den Anreiz für weitere Aktivitäten zu erhalten. \citep[S. 17]{Seelbach.2011}
\newline Nach \citet[S. 250]{Kirschbaum.2008} ist eine sensorisch-spezifische Sättigung eine Sättigung an bestimmten Stoffen, die zu Genüge aufgenommen wurden, z. B. Salz. Für diese Stoffe besteht dann kein Appetit bzw. kein Defizit mehr. Diese Art der Sättigung kann v.a. durch die Anreiztheorie der Motivation erklärt werden. Eine dauerhafte sensorisch-spezifische Sättigung hätte die Folge, dass wir ein Defizit nur einmal befriedigen würden und danach nie wieder, was lebensgefährlich wäre. \citep[S. 17]{Seelbach.2011} 
\newline Nach \citet[S. 17]{Seelbach.2011} birgt diese Erkenntnis eine enorme Bedeutung für Führung und Motivation. Gegen den Einsatz von Incentives, Geschenken oder Zusatzeinkünften in Unternehmen ist nichts einzuwenden. Die Belohnungen werden jedoch allzuoft für das Erreichen eines Ziels in einem zeitlichen Rahmen ausgelobt, wodurch eine Gewöhnung einsetzt und die Wirkung neutralisiert. Beim nächsten Mal ist eine höhere Belohnung nötig, um zu motivieren. Daneben reagiert das Gehirn auf unerwartete Belohnungen intensiver. Die Wirkung teurer und wertvoller Belohnungen wird überschätzt. Nette Worte oder ein Lächeln (also Emotionen) können das Belohnungssystem genauso oder mehr aktivieren als kleinere Überraschungen oder eine unerwartete monetäre Belohnung. Darüberhinaus kann das Belohnungssystem ein altruistisches Verhalten bewirken. Das Sozialverhalten kann dafür sorgen, dass z.B. eine materielle Benachteiligung in akzeptiert wird, wenn dafür z.B. ein \glqq psychologischer Gewinn\grqq für die nötige Befriedigung sorgt. \cite[S. 17]{Seelbach.2011}

\subsubsection{Erinnerungssystem}
\label{sec:Erinnerungssystem}
Die Gesamtheit der bewussten und unbewussten Erlebnisse und Erfahrungen eines Menschen werden im Erinnerungssystem abgelegt und machen ihn zu einem Individuum. Erinnerungen und Erwartungen werden in der selben Hirnregion erzeugt. Darüberhinaus werden Erlebnisse vom Belohnungs- und Emotionssystem mit positiv oder negativ gewertet und in einer Art Rangfolge abgespeichert. 
Das bedeutet, je stärker einzelne Erlebnisse oder Erfahrungen mit positiven oder negativen Emotionen verbunden sind, desto besser behält sie der Mensch in Erinnerung, es erfolgt also eine \glqq privilegierte Abspeicherrung\grqq. \citep[S. 18]{Seelbach.2011}

\subsubsection{Emotionssystem}

Emotionen und Gefühle gehören zum alltäglichen Leben und sind der Psychologie nach  psychische Zustände. Emotionen werden von uns nach außen getragen, während wir Gefühle in uns bewahren. Überraschung, Wut, Angst, Freude und Trauer sind kulturübergreifend. \citep[S. 67]{Nowka.2013}
\citet[S. 17]{Seelbach.2011} betrachtet psychische Zustände aus der neurowissenschaftlichen Sichtweise als chemische Prozesse des Nervensystems. Wie bereits in \ref{sec:DasLimbischeSystem} beschrieben, ist die Amygdala, als Teil des lymbischen Systems, für die Emotionsverarbeitung zuständig. 
Nach \citet[S. 18]{Seelbach.2011} sind Emotionen die Bewertungen von zunächst neutralen Reizen und stehen also zwischen Reiz und Reaktion. Die Reize erreichen die Amygdala und werden dort, je nach den vorhandenen Erinnerungen, bewertet.

\subsubsection{Entscheidungssystem}
Das Entscheidungssystem liegt im präfrontalen Cortex, d.h. im vorderen Bereich des Gehirns. Dort laufen die Informationen aus den drei Systemen, dem Belohnungssystem, dem Emotionssystem und dem Erinnerungssystem zusammen und werden verarbeitet. Der präfrontale Cortex ist also die Kommandozentrale. Dort werden Entscheidungen getroffen unter Berücksichtigung der Normen und Werte und den drei anderen Systemen. Das Entscheidungssystem ist zwar der wichtigster Teil, sogleich aber ohne die andere drei Teile nutzlos. Die Entscheidungen entstehen immer im Wechselspiel zwischen den vier Systemen. \citep[S. 19]{Seelbach.2011}

\subsection{Wirkung von Anreizen}
Die Wirkung von Anreizen ist, wie bereits beschrieben, oft nur durch Betrachtungen und Untersuchungen von \glqq Versuchsgruppen\grqq niedergeschrieben worden. Mit den Erkenntnissen aus den Neurowissenschaften kann heute sehr viel genauer betrachtet werden, wie äußere Anreize auf das Handeln einer Person wirken. Im Folgenden werden die Wirkungen und die Umstände auf Einzelpersonen und Gruppen näher beschrieben.

\subsubsection{Voraussetzungen für die Wirksamkeit eines Anreizes}
Bei der Beurteilung einer Entscheidung werden immer alle aktuell verfügbaren sowie gespeicherten, relevanten Informationen im Gedächtnis hinzugezogen (siehe \ref{sec:Erinnerungssystem}). Somit kann ein Anreiz in verschiedenen Situationen motivierend oder demotivierend wirken, oder gar keine Wirkung zeigen. Je stärker der Anreiz wirkt, desto größer ist seine Wirkung bei der Entscheidungsfindung. Hinzu kommt, dass jeder Mensch verschieden ausgeprägte Bedürfnisse mit entsprechend individuellen Motiven verfolgt. Auch können sich die Motive im Lauf des Lebens stark verändern. \citep[S. 70]{Nowka.2013}
Bezieht man diese Umstände in die Betrachtung mit ein, wird deutlich, wie wichtig es ist, die individuellen Bedürfnisse und Motive einer Person zu berücksichtigen, welche durch einen Anreiz zu einer Handlung bzw. der Unterlassung selbiger, motiviert werden soll. Damit ein Anreiz seine Wirkung entfalten kann, muss ein zugrunde liegendes Bedürfnis existieren, welches der Befriedigung bedarf. \citep[S. 78]{Nowka.2013}

\subsubsection{Zeitlicher Charakter einer Belohnung}
Erhält eine Person eine Belohnung für eine Handlung, wird das Belohnungssystem aktiviert (siehe \ref{sec:Belohnungssystem}). Nach neurowissenschaftlichen Erkenntnissen gibt es dabei jedoch keinen Unterschied, ob die Person intrinsisch oder extrinsisch motiviert ist. \citep[S. 71]{Nowka.2013}
Ebenso führen sowohl eine erwartete wie auch eine unerwartete Belohnung zur selben Reaktion (siehe \ref{sec:Belohnungssystem}). Die Wirkung einer unerwarteten Belohnung ist dabei jedoch deutlich stärker,  bereits die Ankündigung einer Belohnung löst im Belohnungssystem die gleiche Reaktion aus. Bei Erhalt selbiger bleibt das Belohnungssystem jedoch reaktionslos. Das bedeutet, dass das Gehirn nicht zwischen einer unerwarteten und einer angekündigten Belohnung unterscheidet. \citep[S. 71f]{Nowka.2013}
 
Auch der Zeithorizont spielt bei einer Belohnung eine große Rolle. Bei einer Befragung von Probanden, die sich zwischen einer kurzfristigen Belohnung sofort, oder einer großen Belohnung später entscheiden müssen, wählten alle die große Belohnung. Bei einem praktischen Experiment jedoch wählte der überwiegende Teil der Probanden die kurzfristige Belohnung (Marshmallow-Experiment). So bevorzugen die meisten Menschen eine sofortige bzw. zeitnahe Belohnung gegenüber einer späteren, dafür größeren Belohnung. Daraus ergibt sich, dass das Belohnungssystem stärker ist als die Vernunft, die im Entscheidungssystem sitzt. \citep[S. 19]{Seelbach.2011}

\subsubsection{Lerneffekte bei Belohnungen}
Bei mehrfacher Belohnung, insbesondere wenn diese einem regelmäßigen zeitlichen Intervall folgt, besteht ein Lerneffekt. Die Wirkung der Belohnung nimmt ab, da sie als \glqq normal\grqq angesehen wird. Somit muss für die gleiche Wirkung eine höhere Belohnung gewährt werden. \citep[S. 80]{Nowka.2013}
 
Fällt eine zuvor angekündigte Belohnung aus, wird dies vom Gehirn wie ein Verlust behandelt und führt zu Wut und Frustration. Darüber hinaus kommt ein ähnlicher Lerneffekt, wie dem bei häufiger Belohnung, zum Tragen. Dieser sorgt dafür, dass bei einer erneuten Ankündigung einer Belohnung, die Erfahrung aus der zuvor Ausgefallenen, deren Wirkung stark beeinträchtigt. Verstärkt wird die Beeinträchtigung zusätzlich dadurch, dass negative Emotionen wie Wut und Frust, schneller und stärker im Gedächtnis verankert werden. Dies führt zur Priorisierung negativ behafteter Erinnerungen bei der Entscheidungsfindung. Ferner werden negative Erfahrungen meist an andere Personen weiter erzählt. Durch diese Wiederholung der Erinnerung wird deren negative Verankerung zusätzlich gestärkt. \citep[S. 76ff]{Nowka.2013}

\subsubsection{Art der Belohnung}
Bei der Art der Belohnung muss zwischen abstrakter und direkter Belohnung unterschieden werden. Während eine konkrete, greifbare Belohnung direkt als solche wahrgenommen wird, muss eine abstrakte Belohnung zuerst durch einen kognitiven Prozess verarbeitet werden, wodurch sie einen Teil ihrer Wirksamkeit einbüßt. \citep[S. 79f]{Nowka.2013}

\subsubsection{Höhe der Belohnung}
\label{sec:HöheDerBelohnung}
Der Zusammenhang aus Art der Tätigkeit und Höhe der Belohnung hat einen merkbaren Einfluss auf die Leistung einer Person. Bei geistig herausfordernden Tätigkeiten führt eine zu hohe Belohnung zu einer geringeren Leistung, da der Gedanke an die Belohnung ablenkt und sich Anzeichen von Verlustängsten zeigen. Bei primitiven Tätigkeiten hingegen konnte dieser Effekt nicht festgestellt werden. \citep[S. 80f]{Nowka.2013}

\subsubsection{Anreize im sozialen Umfeld}
\label{sec:AnreizeImSozialenUmfeld}
Das Gerechtigkeitsempfinden des Menschen hat bei der Belohnung einer Person innerhalb einer Gruppe großen Einfluss. Generell wird die Belohnung des Einen als Bestrafung des Anderen empfunden. Jeder Mensch strebt nach Gleichbehandlung oder relativer Besserstellung. Das bedeutet, dass ein großer relativer Unterschied der Belohnung innerhalb einer Gruppe als so ungerecht empfunden wird, dass ein genereller Verzicht auf die Belohnung bevorzugt wird. Somit ist die Wirkung der empfundenen Gerechtigkeit größer als die ökonomische Vernunft. Dies beschränkt sich nicht nur auf materielle bzw. monetäre Anreize, sondern erstreckt sich auch auf die soziale Interaktion bzw. Integration innerhalb einer Gruppe. \citep[S. 81ff]{Nowka.2013}
 
Eine mindestens ebenso große Rolle spielt der soziale Status bzw. das zugrunde liegende Anschlussmotiv. Eine positive Änderung des sozialen Status entfaltet im Gehirn eine ähnliche Reaktion wie beispielsweise ein monetärer Anreiz. Insgesamt kommt es jedoch auf Grund des sozialen Drucks generell eher zu einer leichten Minderung der Leistung. \citep[S. 84]{Nowka.2013}
\newpage
\newpage
\section{Handlungsanweisungen}
Im Folgenden werden Maßnahmen, die der Mitarbeitermotivation dienlich sind in drei Punkte unterteilt. Dabei werden sowohl Maßnahmen zur Motivationserhaltung als auch zur Motivationsförderung berücksichtigt. 
Im ersten Teil wird auf den Mitarbeiter als Individuum mit Gefühlen und Bedürfnissen eingegangen. Der zweite Teil konzentriert sich auf das Unternehmen als soziales Umfeld mit mehreren Mitarbeitern in einer hierarchischen Ordnung. Im dritten und letzten Teil liegt der Fokus auf dem Unternehmen als Arbeitsumfeld.

Alle drei Unterpunkte konzentrieren sich dabei auf geistig anspruchsvolle Tätigkeiten. Das bedeutet, dass insbesondere die Maßnahmen der Belohnung und deren Höhe, bei primitiven Tätigkeiten eine andere Wirkung erzielen können, als im Folgenden beschrieben. 

Allen drei Unterpunkten folgt nach der Aufführung der Maßnahmen und ihrer Wirkung eine Wirtschaftlichkeitsbetrachtung. Diese soll dabei unterstützen, ein vertretbares Maß zwischen Wirkung und Wirtschaftlichkeit zu finden. 

\subsection{Mitarbeiter als Individuum}
In den folgenden Abschnitten werden die einzelnen Unterpunkte beleuchtet, die den Mitarbeiter als Individuum betreffen.

\subsubsection{Mitarbeiter als Mensch}
Jeder Mitarbeiter ist in erster Linie ein Mensch, der mit Emotionen und Gefühlen, sowie Bedürfnissen ausgestattet ist. Nach \citet[S. 18]{Seelbach.2011}  stimuliert den Menschen nichts mehr als der Wunsch wahrgenommen zu werden, das umfasst die Aussicht auf soziale und zwischenmenschliche Anerkennung, Wertschätzung und Zuwendung. 
Der Wunsch nach Wahrnehmung kann z.B. damit bedient werden, dass der Mitarbeiter in regelmäßigen Abständen persönlich angesprochen wird. Bereits mit einfachen Worten wie einem Lob oder einem \glqq Danke\grqq d.h. die Wertschätzung geleisteter Arbeit zu hohem Maß motivieren kann. Darüberhinaus läuft die Kommunikation bidirektional, d.h. man sollte dem Mitarbeiter auch zuhören, die Chance geben sich mitteilen zu können. 
Die Bedürfnisse der Menschen sind unterschiedlich, deswegen gilt es eher eine maßgeschneiderte Zuwendung zu versuchen statt einer Gleichbehandlung. 

\subsubsection{Entlohnung}
Die Wertschätzung eines Mitarbeiters drückt sich auch in dessen Gehalt aus. Es ist ein Ausdruck der Entlohnung seiner Fähigkeiten und Zeit. Das bedeutet im Umkehrschluss, dass ein niedriges Gehalt als eine Geringschätzung der Arbeit eines Mitarbeiters oder gar dessen selbst gewertet wird. Generell ist das monatliche Gehalt nicht als Motivator zu sehen, da hierbei sehr schnell der in 3.2 beschriebene Lerneffekt einsetzt und es somit als normal empfunden wird.

Bei der Mitarbeitergewinnung ist das Gehalt ein primärer Anreiz der einem Bewerber angeboten werden kann. Aussicht auf Erfolg besteht dabei nur, wenn der Betrag die Bedürfnisbefriedigung erlaubt und im Vergleich mit anderen Stellen bzw. Unternehmen als gerecht empfunden wird. Geht mit einem Unternehmenswechsel ein merkbarer Gehaltsanstieg einher, so unterliegt auch dieser einem Lerneffekt (siehe 3.2), was dazu führt, dass das höhere Gehalt schon nach kurzer Zeit als normal empfunden wird.

\subsubsection{Tätigkeitsbild}
Der Mensch ist keine Maschine. Diese Aussage ist seit Anfang des 20. Jahrhunderts und spätestens durch Charlie Chaplins Film \glqq Moderne Zeiten\grqq \cite{Chaplin.1936} einer breiten Masse bekannt. 
Mit der Aufnahme einer beruflichen Tätigkeit nach der Ausbildung sind viele Erwartungen und Hoffnungen verknüpft, stellt sie doch einen der wichtigsten Schritte im Leben eines Menschen dar.  
Im Rahmen einer aktuellen Umfrage \citep{Allensbach.2014} unter Studenten wurde unter anderem nach den Erwartungen und Hoffnungen an die eigene zukünftige berufliche Tätigkeit gefragt. Die Antworten der Studenten waren äußerst vielfältig und nicht \glqq primär an materiellen Gratifikaten ausgerichtet\grqq. Mit einem signifikaten Abstand von über 20 \% stehen auf den ersten Plätzen weiche Faktoren. Auf Platz Platz 1 liegt für 73\% der Studenten, der Wunsch nach einem “guten Betriebsklima”. Danach kommt mit 67\% ein “sicherer Arbeitsplatz” und mit 66\%, dass die Tätigkeit kongruent mit den eigenen Fähigkeiten und Neigungen ist. Auf dem vierten Platz kommt für 65\%, dass die Vereinbarkeit zwischen Privatleben und Beruf funktioniert. Erst auf dem 7.Platz kommt ein “hohes Einkommen”.

Die Studie zeigt, wie wichtig eine angemessene Tätigkeit ist. Eine abwechslungsreiche, herausfordernde, aber nicht überfordernde, Tätigkeit, die den eigenen Qualifikationen und Neigungen entspricht, bringt das höchste Motivationspotential mit. 
Die Freiheit bei der Erledigung der Aufgaben kann z.B. durch Zielvorgaben anstatt detaillierte Arbeitsanweisungen erreicht werden. 
Müssen unbeliebte Tätigkeiten erledigt werden, kann mit Inaussichtstellung einer besonders spannenden Tätigkeit motiviert werden. 

\subsubsection{Belohnung}
Zur Leistungserhaltung, aber vor allem zur Leistungssteigerung dienen Anreize in Form von Belohnungen. Generell sollte eine Belohnung dabei in Relation zur erbrachten Leistung stehen. Dabei entfaltet die Ankündigung einer Belohnung dieselbe Wirkung wie die Vergabe einer solchen. Das lässt sich nutzen, um zum Beispiel für die Erledigung einer weniger spannenden Aufgabe, dafür eine umso Interessantere danach in Aussicht zu stellen. 

Bei einer leistungsbezogenen Belohnung spielen zwei Faktoren eine große Rolle für die Wirksamkeit dieser auf die Motivation eines Mitarbeiters. Zum einen muss der Mitarbeiter Einfluss auf die Erreichung der gesetzten Ziele haben. So wird in einem großen Unternehmen der einzelne Sachbearbeiter nur einen sehr geringen Einfluss auf den Jahresgewinn haben. Ein Vertriebsmitarbeiter hingegen, der eine Provision auf seine Verkäufe erhält, hat maßgeblichen Einfluss darauf.
Zweiter wichtiger Faktor ist der Zeitpunkt des Erhalts der Belohnung. Liegt dieser in weiter Ferne, wird der Erhalt als risikoreicher empfunden und somit die Wirkung deutlich geringer ausfallen.

Eine zu hohe Belohnung bewirkt u.U. das Gegenteil der gewünschten Leistungssteigerung (siehe 3.2). Bei jeglicher Form der Belohnung sollte darauf geachtet werden, dass diese bei regelmäßiger Anwendung einem Lerneffekt unterliegen, welcher ihre Wirkung maßgeblich beeinflussen kann. So wird eine jährliche Sonderleistung wie z.B. das in Deutschland weit verbreitete Urlaubsgeld schon nach wenigen Jahren als normal empfunden. Damit wird bereits ein einmaliger Ausfall als Verlust empfunden. Eine Gegenmaßnahme sind unerwartete Belohnungen. Diese unterliegen ebenso dem Lerneffekt, wirken aber grundsätzlich viel stärker (siehe 3.2). 

Direkte Belohnungen wirken in den meisten Fällen deutlich besser als Abstrakte. Geld als bekannteste abstrakte Belohnung hingegen benötigt ein zugrunde liegendes Bedürfnis, welches sich damit befriedigen lässt.

Der größte Demotivator hingegen sind falsche Versprechungen, wenn also eine angekündigte Belohnung ausfällt. Dies führt zu negativen Emotionen wie Wut und Frustration. Vor allem bei mehrfachen Fällen geht damit die gesamte Wirkung einer Belohnungsankündigung verloren (siehe 3.2). Daher sollte in Fällen, bei denen der Erhalt der Belohnung nicht sicher ist, auf eine Ankündigung dieser verzichtet werden, und statt dessen auf den Effekt einer unerwarteten Belohnung im Nachhinein genutzt werden. 

\subsubsection{Wirtschaftlichkeitsbetrachtung}
Bereits mit Wertschätzung und Anerkennung der Leistung eines Mitarbeiters ist oft schon viel erreicht. Ein einfaches Lob und ein Dankeschön kosten kein Geld, haben aber eine nicht zu vernachlässigende Wirkung (siehe 4.1.1). Ebenso kann mit direkten, auf die Bedürfnisse des Mitarbeiters zugeschnittenen Belohnungen in vielen fällen eine größere Wirkung erreicht werden, als mit einer hohen Bohnuszahlung (siehe 4.1.4). Als Beispiel sei hier eine Bahncard genannt, die Mitarbeiter auch privat nutzen kann. 
Natürlich kann nicht jeder Wunsch eines Mitarbeiters berücksichtigt werden. Der notwendige Verwaltungsaufwand wäre unüberschaubar hoch. Jedoch kann wie oben beschrieben mit einfachen, und im Idealfall unbürokratischen, Mitteln, mit weniger finanziellem Aufwand oft mehr erreicht werden. 

Bei den Kosten gilt es neben den kurzfristigen Mehraufwendungen auch die langfristigen Einsparungen zu beachten, dazu zählen z.B. Ausfall von Mitarbeitern durch Krankheit. Nach \citet{Schlolaut.2013} können Unternehmen es sich nicht dauerhaft leisten, ihre Mitarbeiter zu erschöpfen. Der dadurch entstehende wirtschaftliche Schaden für die Unternehmen und für die Gesellschaft ist unübersehbar. Ein weiterer Faktor ist die Vermeidung der “Sinnlosigkeit des eigenen Tuns”. Dazu zählen plötzlich abgebrochene Projekte oder unangemessene Aufgaben, die nichtersichtliche Gründe haben.

\subsection{Unternehmen als soziales Umfeld}
Ein Unternehmen ist ein soziales Umfeld bestehend aus mehreren Mitarbeiter, in welchem eine Vielzahl von sozialen Einflüssen auf die Motivation der Mitarbeiter wirkt. 

\subsubsection{Gerechtigkeitsprinzip}
Die Belohnung von Mitarbeitern in einem Unternehmen ist eine Aufgabe, bei der es mehrere Punkte zu beachten gilt. Die gezielte Belohnung eines Einzelnen kann von Anderen als eine Bestrafung empfunden werden. Dies betrifft jedoch nicht nur Belohnungen im monetären oder materiellen Sinn, sondern erstreckt sich auch auf soziale Anerkennung und Interaktion (siehe 3.2.6). \glqq Für die Führungspraxis bedeutet das: Wer Menschen nachhaltig führen und motivieren will, muss ihnen die Möglichkeit geben, mit anderen zu kooperieren und Beziehungen zu gestalten.\grqq \cite[S. 18]{Seelbach.2011} 

In Deutschland wird überlicherweise nicht über das Gehalt gesprochen. Bei nicht-tarifgebundenen Unternehmen sollte daher ein Gehaltsrahmen definiert werden, der transparente Gehaltsgruppen festlegt. Diese sollten Erfahrung und Tätigkeit eines Mitarbeiters berücksichtigen und somit auf der einen Seite Spielraum für seine individuelle Leistung geben und auf der anderen Seite für ein ausreichendes Maß an Gerechtigkeit zwischen allen Mitarbeitern sorgen. 

\subsubsection{Wirtschaftlichkeitsbetrachtung}
Um die in 4.1.4 genannten Möglichkeiten der individuellen Belohnung in einem transparenten Gehaltsrahmen berücksichtigen zu können, wird im Folgenden anhand des Cafeteria-Modells gezeigt.

Das Cafeteria-Modell wird anhand der drei folgenden Komponenten umgesetzt: einem Budget, einer periodischen Wahlmöglichkeit und einem Wahlangebot.
Der Vorteil ist, dass das Unternehmen ein festes Budget einplanen kann, das pro Periode für dieses Anreizsystem zur Verfügung steht. Der Mitarbeiter kann seine persönlichen Bedürfnisse gezielt befriedigen, ohne dass eine zeitaufwändige Analyse durch eine dritte Person erfolgt. Dadurch entsteht ein hoher Wirkungsgrad. Das dem Mitarbeiter zur Verfügung stehende Budget kann durch besondere Leistungen erweitert werden.
Gleichzeitig wird die Gerechtigkeit durch einen allgemein verbindlichen Rahmen gewährleistet. Das Zusammengehörigkeitsgefühl kann dadurch auch gestärkt werden, indem z.B. gemeinsam Aktivitäten ausgeübt werden können. Ein Nachteil für das Unternehmen ist beim Cafeteria-Modell der zusätzliche Verwaltungsaufwand. Auf der einen Seite müssen die Mitarbeiter nicht mehr analysiert werden, auf der anderen Seite ist jedoch das Management der einzelnen \glqq Wahlangebote\grqq mit einem gewissen Aufwand verbunden (z.B. Vertragslaufzeiten der Bahncard usw.).
Je nach Land, z.B. in Deutschland, ist auch die steuerrechtliche Grundlage kompliziert. 
Davon abgesehen können über das Cafeteria-Modell nur monetär abbildbare Werte angeboten werden. \cite[S. 55ff]{Nowka.2013}

\subsection{Unternehmen als Arbeitsumfeld}
Das Unternehmen bietet seinen Mitarbeitern einen Arbeitsplatz zur Erbringung ihrer Arbeitsleistung. 

\subsubsection{Anreize im Arbeitsumfeld}
Die Gestaltung des Arbeitsumfelds bietet vielfältige Möglichkeiten zur Anreizgestaltung. Angefangen bei der individuellen Arbeitsplatzausstattung über Gemeinschaftsräume wie Pausenraum bis zu Wohlfühlfaktoren wie kostenlosen Heiß- und Kaltgetränken sind hier einem Unternehmen keine Grenzen gesetzt. Ziel dieser Maßnahmen ist es, den Aufenthalt der Mitarbeiter am Arbeitsplatz so angenehm wie möglich zu gestalten. Dabei sind auch Kooperationen mit anderen Unternehmen plausibel, wie das Beispiel unternehmensübergreifender Kindertagesstätten zeigt. 

All diese Anreize können bei der Mitarbeitergewinnung als zusätzliche Vorteile angeführt werden. Sie fördern die Motivation  Längerfristige Maßnahmen wie Betriebskindertagesstätten fördern darüber hinaus die Mitarbeiterbindung. 

In erster Linie bei der Arbeitsplatzausstattung gilt jedoch auch ein Mindestmaß, welches Mitarbeiter zur Erledigung ihrer Arbeit als Voraussetzung sehen. Wird dieses nicht erfüllt, so entsteht sehr schnell Frustration. Ebenso unterliegen alle oben genannten Formen von Anreizen einem Lern- bzw. Gewöhnungseffekt. Sind die Mitarbeiter es gewohnt, kostenlose Getränke am Arbeitsort zu erhalten, wird ein Wegfallen dieser Leistung als Verlust wahrgenommen und führt somit ebenfalls zu Demotivation und fördert Frustration. Daher sollte bei Einführung dieser Maßnahmen darauf geachtet werden, dass diese in erster Linie nicht der kurzfristigen Förderung der Mitarbeitermotivation dienen, sondern viel mehr dem langfristigen Erhalt der Motivation.

\subsubsection{Wirtschaftlichkeitsbetrachtung}
Generell ist bei der Arbeitsumgebung darauf zu achten, dass der als üblich erachtete Mindeststandard erfüllt werden kann, damit diese nicht demotivierend wirkt. Eine Erhöhung des Standards bringt zwar die bereits beschäftigten Mitarbeiter nur einen kurzzeitigen Motivationsschub, kann jedoch als positiver Anreiz für die Mitarbeitergewinnung genutzt werden. Wichtig ist hier sich an den Standards der eigenen Branche zu orientieren. Werden Maßnahmen wie kostenlose Getränke für alle Mitarbeiter eingeführt, sollte bei der Einführung auf jeden Fall darauf geachtet werden, dass eine Abschaffung dieser mit einer hohen Demotivation der Mitarbeiter einhergeht.
\newpage
\section{Fazit \& Ausblick} % (fold)

\subsection*{Fazit}
Alle klassischen Motivationstheorien basieren ausschließlich auf empirischer Untersuchung. Sie sind also ein Beobachten von außen. Die neurowissenschaftliche Forschung liefert Erkenntnisse darüber, welche Vorgänge im Innern des Gehirns ablaufen. Damit lässt sich die Wirkungsweise von Anreizen auf das menschliche Verhalten deutlich besser beschreiben. 

Trotzdem kann aufgrund des individuellen, situativen Charakters von Anreizen keine Garantie gegeben werden, wie ein bestimmter Anreiz bei einer bestimmten Person wirkt. Zu komplex ist die Anzahl der Einflussfaktoren. Unabhängig davon lässt sich aus dem Wissen der generellen Wirkungsweise von Anreizen in Kombination mit Kenntniss einer Person, bereits sehr gut vorhersehen, ob, und wie, ein bestimmter Anreiz auf diese wirkt.

Unternehmen haben das Ziel, mit möglichst einfachen und zahlenmäßig wenigen Mitteln, möglichst viele Mitarbeiter auf einmal motivieren zu können, ohne dabei die Motivation anderer einzuschränken. In Kapitel 4 konnte ein kurzer Überblick gegeben werden, was sie bei der Einführung bzw. Gestaltung von Anreizsystem zur Mitarbeitermotivation beachten sollten. Damit haben sie die Möglichkeit, ihre Maßnahmen zielgerichteter einzusetzen, und vor allem die Wirkungsweisen von Demotivatoren und Gewöhnungs- bzw. Lerneffekten zu berücksichtigen.

Zusammengefasst kann gesagt werden, dass oft die kleinen Dinge sehr viel bewirken können. Wichtigster Faktor ist dabei die Kommunikation. Wer sein Gegenüber, der zu mehr Leistung motiviert werden soll, nicht gut kennt, kann auch nicht wissen, was diesen zu besonderer Leistung anstiftet.





\subsection*{Ausblick}
Diese Arbeit spiegelt den aktuellen Stand der neurowissenschaftliche Forschung wieder. Zukünftige, bahnbrechende Entdeckungen sind dabei aufgrund der Komplexität des menschlichen Gehirns nicht auszuschließen. 



\glqq Aus neurowissenschaftlicher Sicht ist der \glqq homo oeconomicus\grqq in seiner reinen Ausprägung nicht haltbar. Das geflügelte Wort von \glqq der Bezahlung als Hygienefaktor\grqq zeigt, dass das Wissen darum in der Praxis eigentlich schon längst angekommen ist. Die Gehirnforschung liefert in diesem Punkt die Beweise. Die Bezahlung muss stimmen, damit sich Mitarbeiter fair behandelt fühlen. Begeisterung und Engagement wachsen jedoch auf einem anderen Boden. Eine faire und kooperative Zusammenarbeit rückt in den Blickpunkt. Es lohnt sich also in ein gutes Betriebsklima zu investieren.\grqq [Seelbach, S. 19]



Allerdings sollten lobende Worte und Belohnungen maßvoll und unregelmäßig verwendet werden, um einen Gewöhnungseffekt zu vermeiden.\grqq [Nowka, S. 88]


Der aktuelle Stand der Technik ist nur der erste Schritt bei der Erforschung des menschlichen Gehirns. Die Visualisierung der Aktivitäten innerhalb des menschlichen Körpers ist ein großer Schritt im Vergleich zur äußerlichen Beobachtung des Verhaltens von Probanden. 

Die technische Grenze des fMRT ist, dass nicht direkt die Nervenzellaktivität sondern der sich ändernde Sauerstoffgehalt im Blutfluss. [Weber, S. 50]. Besonders schnelle Vorgänge ohne Sauerstoffbedarf nicht identifizierbar [Nowka, S. 96]

Darüberhinaus ist die Komplexität des menschlichen Gehirns unvorstellbar hoch und noch bei Weitem nicht gänzlich erforscht. Die vollständige Erforschung wird, je nach Quelle, als zweifelhaft angesehen und selbst wenn, dann existiert immer noch der Faktor Mensch, dessen Verhaltens- und Denkweisen als Individuum, auch mit der Erforschung nicht begründet werden könnten. [Nowka, S. 96]

Missdeutung der Zuständigkeit von Gehirnarealen

-Individuum bleibt Individuum d.h. generelle Aussagen nicht treffbar sonder nur tendenziell möglich
\newline \glqq Für eine wirkungsvolle Anreizgestaltung im Unternehmen muss die gesamte Bandbreite an möglichen Anreizen genauso beachtet werden, wie die jeweilige Betriebssituation und die individuellen Bedürfnisse der Mitarbeiter.\grqq \citep[S. 54]{Nowka.2013}

% % % % % % % % % % % % % % % % % % % % % % % % % % % % % % % % % % % % % % % % % % % % % % % % % % % % % % % % % % % % % % % %


\small
\begin{table}[!ht]
\caption{Hauptkategorien der Evaluationskriterien}
\label{Hauptkategorien1}
\begin{tabularx}{\textwidth}[b]{|p{5mm} p{40mm}|X|}
\hline
\rowcolor{black!10} \multicolumn{2}{|l|}{\normalsize\textbf{Kategorie}}  & \normalsize\textbf{Beschreibung}  \tabularnewline
\hline
1 & \raggedright\textit{Modellierung} & Unter dem Aspekt Modellierung werden alle Funktionalitaeten zusammen gefasst, die den Benutzer dabei unterstuetzen, Prozessmodelle zu erstellen und zu beschreiben. \tabularnewline
\hline
2 & \raggedright\textit{Implementierung} & Umfasst alle Funktionalitaeten, welche den Nutzer bei der Umsetzung der Automatisierung eines Prozesses unterstuetzen.  \tabularnewline
\hline
3 & \raggedright\textit{Prozessausfuehrung} & Beinhaltet alle den Nutzer waehrend der Ausfuehrung eines Prozesses direkt unterstuetzenden Funktionalitaeten. \tabularnewline
\hline
4 & \raggedright\textit{Steuerung und Ueberwachung} & Fasst alle Funktionalitaeten zusammen, welche dem Benutzer eine Uebersicht und Steuerungsmoeglichkeiten ueber Prozesse und System zur Laufzeit ermoeglichen. \tabularnewline
\hline
5 & \raggedright\textit{Analyse} & Buendelt die Funktionalitaeten, die den Benutzer bei der Analyse von Prozessen und deren Performance unterstuetzen. \tabularnewline
\hline
6 & \raggedright\textit{Allgemeine Software Anforderungen} & Umfasst die Kriterien, welche fuer komplexe Unternehmenssoftware im Allgemeinen gueltig sind. \tabularnewline
\hline
\end{tabularx}
\end{table}
\normalsize


\begin{table}[h]
\begin{tabularx}{\textwidth}{XXX}
\hline

Deutscher Name (wissensch. Name)              & Strukturen                                    & Funktion                                                                             \\ \hline
Nachhirn                                      & Verlaengertes Rueckenmark (Medulla oblongata) & Vegetatives (unbewusstes) Steuerzentrum (Atmung, Kreislauf, Verdauung, Reflexe, ...) \\ \hline
                                              & Kleinhirn (Cerebellum)                        & Gleichgewicht, unbewusstes Feintuning von Bewegungen                                 \\ \cline{2-3} 
\multirow{-2}{*}{Hinterhirn (Myelencephalon)} & Bruecke (Pons)                                & Nervenfasern, die zum Kleinhirn ziehen                                               \\ \hline
Mittelhirn (Mesencephalon)                    & Kerne von Hirnnerven und Neurotransmittern    & Bildung zahlreicher Neurotransmitter, Steuerung der Augenbewegungen                  \\ \hline
Zwischenhirn (Diencephalon)                   & Thalamus                                      & Schaltstelle fast aller Nervenfasern, die zur Grosshirnrinde ziehen                   \\ \cline{2-3} 
                                              & Hypothalamus                                  & Steuerung des Hormonhaushaltes                                                       \\ \hline
Gross-/Endhirn (Telencephalon)                 & Grosshirnrinde (Cortex)                        & siehe 2.4                                                                            \\ \cline{2-3} 
                                              & Basalganglien                                 & unbewusste Steuerung von Bewegungen                                                  \\ \cline{2-3} 
                                              & Innere Kapsel                                 & Ansammlung von Nervenfasern, die zur Grosshirnrinde ziehen                            \\ \cline{2-3} 
                                              & Balken                                        & Faserbuendel, das die beiden Grosshirnhaelften verbindet                              \\ \hline
\end{tabularx}
\end{table}






\bibliography{bibliography}

\end{document}