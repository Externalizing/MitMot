\newpage
\section{Fazit \& Ausblick} % (fold)

\subsection*{Fazit}
Alle klassischen Motivationstheorien basieren ausschließlich auf empirischer Untersuchung. Sie sind also ein Beobachten von außen. Die neurowissenschaftliche Forschung liefert Erkenntnisse darüber, welche Vorgänge im Innern des Gehirns ablaufen. Damit lässt sich die Wirkungsweise von Anreizen auf das menschliche Verhalten deutlich besser beschreiben. 

Trotzdem kann aufgrund des individuellen, situativen Charakters von Anreizen keine Garantie gegeben werden, wie ein bestimmter Anreiz bei einer bestimmten Person wirkt. Zu komplex ist die Anzahl der Einflussfaktoren. Unabhängig davon lässt sich aus dem Wissen der generellen Wirkungsweise von Anreizen in Kombination mit Kenntniss einer Person, bereits sehr gut vorhersehen, ob, und wie, ein bestimmter Anreiz auf diese wirkt.

Unternehmen haben das Ziel, mit möglichst einfachen und zahlenmäßig wenigen Mitteln, möglichst viele Mitarbeiter auf einmal motivieren zu können, ohne dabei die Motivation anderer einzuschränken. In Kapitel 4 konnte ein kurzer Überblick gegeben werden, was sie bei der Einführung bzw. Gestaltung von Anreizsystem zur Mitarbeitermotivation beachten sollten. Damit haben sie die Möglichkeit, ihre Maßnahmen zielgerichteter einzusetzen, und vor allem die Wirkungsweisen von Demotivatoren und Gewöhnungs- bzw. Lerneffekten zu berücksichtigen.

Zusammengefasst kann gesagt werden, dass oft die kleinen Dinge sehr viel bewirken können. Wichtigster Faktor ist dabei die Kommunikation. Wer sein Gegenüber, der zu mehr Leistung motiviert werden soll, nicht gut kennt, kann auch nicht wissen, was diesen zu besonderer Leistung anstiftet.





\subsection*{Ausblick}
Diese Arbeit spiegelt den aktuellen Stand der neurowissenschaftliche Forschung wieder. Zukünftige, bahnbrechende Entdeckungen sind dabei aufgrund der Komplexität des menschlichen Gehirns nicht auszuschließen. 



\glqq Aus neurowissenschaftlicher Sicht ist der \glqq homo oeconomicus\grqq in seiner reinen Ausprägung nicht haltbar. Das geflügelte Wort von \glqq der Bezahlung als Hygienefaktor\grqq zeigt, dass das Wissen darum in der Praxis eigentlich schon längst angekommen ist. Die Gehirnforschung liefert in diesem Punkt die Beweise. Die Bezahlung muss stimmen, damit sich Mitarbeiter fair behandelt fühlen. Begeisterung und Engagement wachsen jedoch auf einem anderen Boden. Eine faire und kooperative Zusammenarbeit rückt in den Blickpunkt. Es lohnt sich also in ein gutes Betriebsklima zu investieren.\grqq [Seelbach, S. 19]



Allerdings sollten lobende Worte und Belohnungen maßvoll und unregelmäßig verwendet werden, um einen Gewöhnungseffekt zu vermeiden.\grqq [Nowka, S. 88]


Der aktuelle Stand der Technik ist nur der erste Schritt bei der Erforschung des menschlichen Gehirns. Die Visualisierung der Aktivitäten innerhalb des menschlichen Körpers ist ein großer Schritt im Vergleich zur äußerlichen Beobachtung des Verhaltens von Probanden. 

Die technische Grenze des fMRT ist, dass nicht direkt die Nervenzellaktivität sondern der sich ändernde Sauerstoffgehalt im Blutfluss. [Weber, S. 50]. Besonders schnelle Vorgänge ohne Sauerstoffbedarf nicht identifizierbar [Nowka, S. 96]

Darüberhinaus ist die Komplexität des menschlichen Gehirns unvorstellbar hoch und noch bei Weitem nicht gänzlich erforscht. Die vollständige Erforschung wird, je nach Quelle, als zweifelhaft angesehen und selbst wenn, dann existiert immer noch der Faktor Mensch, dessen Verhaltens- und Denkweisen als Individuum, auch mit der Erforschung nicht begründet werden könnten. [Nowka, S. 96]

Missdeutung der Zuständigkeit von Gehirnarealen

-Individuum bleibt Individuum d.h. generelle Aussagen nicht treffbar sonder nur tendenziell möglich
\newline \glqq Für eine wirkungsvolle Anreizgestaltung im Unternehmen muss die gesamte Bandbreite an möglichen Anreizen genauso beachtet werden, wie die jeweilige Betriebssituation und die individuellen Bedürfnisse der Mitarbeiter.\grqq \citep[S. 54]{Nowka.2013}

% % % % % % % % % % % % % % % % % % % % % % % % % % % % % % % % % % % % % % % % % % % % % % % % % % % % % % % % % % % % % % % %


\small
\begin{table}[!ht]
\caption{Hauptkategorien der Evaluationskriterien}
\label{Hauptkategorien1}
\begin{tabularx}{\textwidth}[b]{|p{5mm} p{40mm}|X|}
\hline
\rowcolor{black!10} \multicolumn{2}{|l|}{\normalsize\textbf{Kategorie}}  & \normalsize\textbf{Beschreibung}  \tabularnewline
\hline
1 & \raggedright\textit{Modellierung} & Unter dem Aspekt Modellierung werden alle Funktionalitaeten zusammen gefasst, die den Benutzer dabei unterstuetzen, Prozessmodelle zu erstellen und zu beschreiben. \tabularnewline
\hline
2 & \raggedright\textit{Implementierung} & Umfasst alle Funktionalitaeten, welche den Nutzer bei der Umsetzung der Automatisierung eines Prozesses unterstuetzen.  \tabularnewline
\hline
3 & \raggedright\textit{Prozessausfuehrung} & Beinhaltet alle den Nutzer waehrend der Ausfuehrung eines Prozesses direkt unterstuetzenden Funktionalitaeten. \tabularnewline
\hline
4 & \raggedright\textit{Steuerung und Ueberwachung} & Fasst alle Funktionalitaeten zusammen, welche dem Benutzer eine Uebersicht und Steuerungsmoeglichkeiten ueber Prozesse und System zur Laufzeit ermoeglichen. \tabularnewline
\hline
5 & \raggedright\textit{Analyse} & Buendelt die Funktionalitaeten, die den Benutzer bei der Analyse von Prozessen und deren Performance unterstuetzen. \tabularnewline
\hline
6 & \raggedright\textit{Allgemeine Software Anforderungen} & Umfasst die Kriterien, welche fuer komplexe Unternehmenssoftware im Allgemeinen gueltig sind. \tabularnewline
\hline
\end{tabularx}
\end{table}
\normalsize


\begin{table}[h]
\begin{tabularx}{\textwidth}{XXX}
\hline

Deutscher Name (wissensch. Name)              & Strukturen                                    & Funktion                                                                             \\ \hline
Nachhirn                                      & Verlaengertes Rueckenmark (Medulla oblongata) & Vegetatives (unbewusstes) Steuerzentrum (Atmung, Kreislauf, Verdauung, Reflexe, ...) \\ \hline
                                              & Kleinhirn (Cerebellum)                        & Gleichgewicht, unbewusstes Feintuning von Bewegungen                                 \\ \cline{2-3} 
\multirow{-2}{*}{Hinterhirn (Myelencephalon)} & Bruecke (Pons)                                & Nervenfasern, die zum Kleinhirn ziehen                                               \\ \hline
Mittelhirn (Mesencephalon)                    & Kerne von Hirnnerven und Neurotransmittern    & Bildung zahlreicher Neurotransmitter, Steuerung der Augenbewegungen                  \\ \hline
Zwischenhirn (Diencephalon)                   & Thalamus                                      & Schaltstelle fast aller Nervenfasern, die zur Grosshirnrinde ziehen                   \\ \cline{2-3} 
                                              & Hypothalamus                                  & Steuerung des Hormonhaushaltes                                                       \\ \hline
Gross-/Endhirn (Telencephalon)                 & Grosshirnrinde (Cortex)                        & siehe 2.4                                                                            \\ \cline{2-3} 
                                              & Basalganglien                                 & unbewusste Steuerung von Bewegungen                                                  \\ \cline{2-3} 
                                              & Innere Kapsel                                 & Ansammlung von Nervenfasern, die zur Grosshirnrinde ziehen                            \\ \cline{2-3} 
                                              & Balken                                        & Faserbuendel, das die beiden Grosshirnhaelften verbindet                              \\ \hline
\end{tabularx}
\end{table}



