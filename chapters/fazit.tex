\newpage
\section{Fazit \& Ausblick} % (fold)

\subsection*{Fazit}
Alle klassischen Motivationstheorien basieren ausschließlich auf empirischer Untersuchung. Sie sind also ein Beobachten von außen. Die neurowissenschaftliche Forschung liefert Erkenntnisse darüber, welche Vorgänge im Innern des Gehirns ablaufen. Damit lässt sich die Wirkungsweise von Anreizen auf das menschliche Verhalten deutlich besser beschreiben. 

Trotzdem kann aufgrund des individuellen, situativen Charakters von Anreizen keine Garantie gegeben werden, wie ein bestimmter Anreiz bei einer bestimmten Person wirkt. Zu komplex ist die Anzahl der Einflussfaktoren. Unabhängig davon lässt sich aus dem Wissen der generellen Wirkungsweise von Anreizen in Kombination mit Kenntniss einer Person, bereits sehr gut vorhersehen, ob, und wie, ein bestimmter Anreiz auf diese wirkt.

Unternehmen haben das Ziel, mit möglichst einfachen und zahlenmäßig wenigen Mitteln, möglichst viele Mitarbeiter auf einmal motivieren zu können, ohne dabei die Motivation anderer einzuschränken. In Kapitel 4 konnte ein kurzer Überblick gegeben werden, was sie bei der Einführung bzw. Gestaltung von Anreizsystem zur Mitarbeitermotivation beachten sollten. Damit haben sie die Möglichkeit, ihre Maßnahmen zielgerichteter einzusetzen, und vor allem die Wirkungsweisen von Demotivatoren und Gewöhnungs- bzw. Lerneffekten zu berücksichtigen.

Zusammengefasst kann gesagt werden, dass oft die kleinen Dinge sehr viel bewirken können. Wichtigster Faktor ist dabei die Kommunikation. Wer sein Gegenüber, der zu mehr Leistung motiviert werden soll, nicht gut kennt, kann auch nicht wissen, was diesen zu besonderer Leistung anstiftet.

\subsection*{Ausblick}
Diese Arbeit spiegelt den aktuellen Stand der neurowissenschaftlichen Forschung wieder. Zukünftige, bahnbrechende Entdeckungen sind dabei aufgrund der Komplexität des menschlichen Gehirns nicht auszuschließen. 

In der Zukunft werden Anreize u.U. nicht mehr als Maßnahmen von außen gesteuert, sondern durch die Wirkung von Psychopharmaka deutlich zielgerichteter zur Motiationsförderung eingesetzt. Der Spektrum zeigt Möglichkeiten, die mit diesen neuen Erkenntnissen der Forschung, vielleicht bereits in naher Zukunft möglich sein werden. So wäre es durchaus Vorstellbar, dass Psychopharmaka zur Stimulierung des jeweils angepassten bzw. notwendigen Gemütszustandes gebrauch finden. Eben wie die passende Krawatte oder der aktuell angesagten Partylaune. \citep{Spektrum.2014}




