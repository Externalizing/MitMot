\newpage
\section{Fazit \& Ausblick} % (fold)

\subsection{Fazit}
Der aktuelle Stand der Technik ist nur der erste Schritt bei der Erforschung des Gehirns. Die Visualisierung der Aktivitäten innerhalb des menschlichen Körpers ist ein großer Schritt im Vergleich zur äußerlichen Beobachtung des Verhaltens von Probanden. 

Die technische Grenze des fMRT ist, dass nicht direkt die Nervenzellaktivität sondern der sich ändernde Sauerstoffgehalt im Blutfluss. [Weber, S. 50]. Besonders schnelle Vorgänge ohne Sauerstoffbedarf nicht identifizierbar [Nowka, S. 96]

Darüberhinaus ist die Komplexität des menschlichen Gehirns unvorstellbar und noch bei Weitem nicht auch nur gänzlich erforscht. Die vollständige Erforschung wird, je nach Quelle, als zweifelhaft angesehen und selbst wenn, dann existiert immer noch der Faktor Mensch, der als Individuum, dessen Verhaltens- und Denkweisen auch mit der Erforschung nicht begründet werden könnten. [Nowka, S. 96]

Missdeutung der Zuständigkeit von Gehirnarealen

-Individuum bleibt Individuum d.h. generelle Aussagen nicht treffbar sonder nur tendenziell möglich

\subsection{Ausblick}
Der aktuellen Stand der Forschung wird nur wiedergespiegelt. Zukünftige bahnbrechende Entdeckungen sind aufgrund der Komplexität des menschlichen Gehirns nicht auszuschließen. 

Motivationstheorien meist auf empirischer Basis nachgewiesen, können keinen Anspruch auf Abdeckung aller Fälle erheben

\glqq Aus neurowissenschaftlicher Sicht ist der \glqq homo oeconomicus\grqq in seiner reinen Ausprägung nicht haltbar. Das geflügelte Wort von \glqq der Bezahlung als Hygienefaktor\grqq zeigt, dass das Wissen darum in der Praxis eigentlich schon längst angekommen ist. Die Gehirnforschung liefert in diesem Punkt die Beweise. Die Bezahlung muss stimmen, damit sich Mitarbeiter fair behandelt fühlen. Begeisterung und Engagement wachsen jedoch auf einem anderen Boden. Eine faire und kooperative Zusammenarbeit rückt in den Blickpunkt. Es lohnt sich also in ein gutes Betriebsklima zu investieren.\grqq [Seelbach, S. 19]

\glqq Für eine wirkungsvolle Anreizgestaltung im Unternehmen muss die gesamte Bandbreite an möglichen Anreizen genauso beachtet werden, wie die jeweilige Betriebssituation und die individuellen Bedürfnisse der Mitarbeiter.\grqq [Nowka, S. 54]

Allerdings sollten lobende Worte und Belohnungen maßvoll und unregelmäßig verwendet werden, um einen Gewöhnungseffekt zu vermeiden.\grqq [Nowka, S. 88]