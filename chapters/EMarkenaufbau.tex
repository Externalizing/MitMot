\section{Aktueller Stand der Forschung}
Eine Person zu motivieren bedeutet Einfluss auf Ihre zukünftigen Entscheidungen zu nehmen. Jeder Entscheidung liegt mindestens ein Motiv zugrunde. Unklar ist dabei jedoch, wie hoch der Einfluss auf die Entscheidungsfindung ist. Früher wurden zur Untersuchung der Wirkung oft Versuchspersonen mit Anreizen motiviert und ihr Verhalten betrachtet. Der im Gehirn ablaufende Prozess zwischen Wahrnehmung des Anreizes und Entscheidungsfindung konnte nicht betrachtet werden. Heute ist dies mit den modernen Methoden der Neurowissenschaften möglich. \citep[S. 59]{Nowka.2013}
Die Neurowissenschaft bietet eine Möglichkeit, den Entscheidungsfindungsprozess genauer untersuchen zu können. Dabei gilt es zu berücksichtigen, dass jede Entscheidung von Informationen abhängt, die zu diesem Zeitpunkt im Gehirn eines Individuums gespeichert sind oder gerade verarbeitet werden. \citep[S. 60]{Nowka.2013}

\subsection{Neurowissenschaftliche Gehirnsysteme}
Die Wirkung eines Anreizes lässt sich heutzutage Mittels der Methoden der Neurowissenschaften (siehe \ref{sec:NeurobiologischeGrundlagen}) nachweisen, womit es möglichist, das Potenzial verschiedener Anreize aufzuzeigen. Die folgenden funktionalen Systeme sind im Gehirn vor allem für Führung und Motivation verantwortlich: 

\begin{APAitemize}
\item das Belohnungssystem
\item das Erinnerungssystem
\item das Emotionssystem
\item das Entscheidungssystem
\end{APAitemize}

\glqq Diese vier Systeme sind wesentlich dafür verantwortlich, ob und in welchem Umfang Führung und Motivation gelingen.\grqq \citep[S. 16]{Seelbach.2011}

\subsubsection{Belohnungssystem}
\label{sec:Belohnungssystem}
Das Belohnunggsystem umfasst unter anderem das, bereits erwähnte, ventrale tegmentale Areal (VTA) des Mittelhirns und den präfrontalen Cortex (siehe \ref{sec:dasGehirn}) und ist nicht dazu vorgesehen, dauerhaftes Wohlbefinden zu erzeugen. Vielmehr geht es bei den Prozessen im Belohnungszentrum darum, durch einen Wechsel zwischen Aktivierung und Deaktivierung den Anreiz für weitere Aktivitäten zu erhalten. \citep[S. 17]{Seelbach.2011}
\newline Nach \citet[S. 250]{Kirschbaum.2008} ist eine sensorisch-spezifische Sättigung eine Sättigung an bestimmten Stoffen, die zu Genüge aufgenommen wurden, z. B. Salz. Für diese Stoffe besteht dann kein Appetit bzw. kein Defizit mehr. Diese Art der Sättigung kann v.a. durch die Anreiztheorie der Motivation erklärt werden. Eine dauerhafte sensorisch-spezifische Sättigung hätte die Folge, dass wir ein Defizit nur einmal befriedigen würden und danach nie wieder, was lebensgefährlich wäre. \citep[S. 17]{Seelbach.2011} 
\newline Nach \citet[S. 17]{Seelbach.2011} birgt diese Erkenntnis eine enorme Bedeutung für Führung und Motivation. Gegen den Einsatz von Incentives, Geschenken oder Zusatzeinkünften in Unternehmen ist nichts einzuwenden. Die Belohnungen werden jedoch allzuoft für das Erreichen eines Ziels in einem zeitlichen Rahmen ausgelobt, wodurch eine Gewöhnung einsetzt und die Wirkung neutralisiert. Beim nächsten Mal ist eine höhere Belohnung nötig, um zu motivieren. Daneben reagiert das Gehirn auf unerwartete Belohnungen intensiver. Die Wirkung teurer und wertvoller Belohnungen wird überschätzt. Nette Worte oder ein Lächeln (also Emotionen) können das Belohnungssystem genauso oder mehr aktivieren als kleinere Überraschungen oder eine unerwartete monetäre Belohnung. Darüberhinaus kann das Belohnungssystem ein altruistisches Verhalten bewirken. Das Sozialverhalten kann dafür sorgen, dass z.B. eine materielle Benachteiligung in akzeptiert wird, wenn dafür z.B. ein \glqq psychologischer Gewinn\grqq für die nötige Befriedigung sorgt. \cite[S. 17]{Seelbach.2011}

\subsubsection{Erinnerungssystem}
\label{sec:Erinnerungssystem}
Die Gesamtheit der bewussten und unbewussten Erlebnisse und Erfahrungen eines Menschen werden im Erinnerungssystem abgelegt und machen ihn zu einem Individuum. Erinnerungen und Erwartungen werden in der selben Hirnregion erzeugt. Darüberhinaus werden Erlebnisse vom Belohnungs- und Emotionssystem mit positiv oder negativ gewertet und in einer Art Rangfolge abgespeichert. 
Das bedeutet, je stärker einzelne Erlebnisse oder Erfahrungen mit positiven oder negativen Emotionen verbunden sind, desto besser behält sie der Mensch in Erinnerung, es erfolgt also eine \glqq privilegierte Abspeicherrung\grqq. \citep[S. 18]{Seelbach.2011}

\subsubsection{Emotionssystem}

Emotionen und Gefühle gehören zum alltäglichen Leben und sind der Psychologie nach  psychische Zustände. Emotionen werden von uns nach außen getragen, während wir Gefühle in uns bewahren. Überraschung, Wut, Angst, Freude und Trauer sind kulturübergreifend. \citep[S. 67]{Nowka.2013}
\citet[S. 17]{Seelbach.2011} betrachtet psychische Zustände aus der neurowissenschaftlichen Sichtweise als chemische Prozesse des Nervensystems. Wie bereits in \ref{sec:DasLimbischeSystem} beschrieben, ist die Amygdala, als Teil des lymbischen Systems, für die Emotionsverarbeitung zuständig. 
Nach \citet[S. 18]{Seelbach.2011} sind Emotionen die Bewertungen von zunächst neutralen Reizen und stehen also zwischen Reiz und Reaktion. Die Reize erreichen die Amygdala und werden dort, je nach den vorhandenen Erinnerungen, bewertet.

\subsubsection{Entscheidungssystem}
Das Entscheidungssystem liegt im präfrontalen Cortex, d.h. im vorderen Bereich des Gehirns. Dort laufen die Informationen aus den drei Systemen, dem Belohnungssystem, dem Emotionssystem und dem Erinnerungssystem zusammen und werden verarbeitet. Der präfrontale Cortex ist also die Kommandozentrale. Dort werden Entscheidungen getroffen unter Berücksichtigung der Normen und Werte und den drei anderen Systemen. Das Entscheidungssystem ist zwar der wichtigster Teil, sogleich aber ohne die andere drei Teile nutzlos. Die Entscheidungen entstehen immer im Wechselspiel zwischen den vier Systemen. \citep[S. 19]{Seelbach.2011}

\subsection{Wirkung von Anreizen}
Die Wirkung von Anreizen ist, wie bereits beschrieben, oft nur durch Betrachtungen und Untersuchungen von \glqq Versuchsgruppen\grqq niedergeschrieben worden. Mit den Erkenntnissen aus den Neurowissenschaften kann heute sehr viel genauer betrachtet werden, wie äußere Anreize auf das Handeln einer Person wirken. Im Folgenden werden die Wirkungen und die Umstände auf Einzelpersonen und Gruppen näher beschrieben.

\subsubsection{Voraussetzungen für die Wirksamkeit eines Anreizes}
Bei der Beurteilung einer Entscheidung werden immer alle aktuell verfügbaren sowie gespeicherten, relevanten Informationen im Gedächtnis hinzugezogen (siehe \ref{sec:Erinnerungssystem}). Somit kann ein Anreiz in verschiedenen Situationen motivierend oder demotivierend wirken, oder gar keine Wirkung zeigen. Je stärker der Anreiz wirkt, desto größer ist seine Wirkung bei der Entscheidungsfindung. Hinzu kommt, dass jeder Mensch verschieden ausgeprägte Bedürfnisse mit entsprechend individuellen Motiven verfolgt. Auch können sich die Motive im Lauf des Lebens stark verändern. \citep[S. 70]{Nowka.2013}
Bezieht man diese Umstände in die Betrachtung mit ein, wird deutlich, wie wichtig es ist, die individuellen Bedürfnisse und Motive einer Person zu berücksichtigen, welche durch einen Anreiz zu einer Handlung bzw. der Unterlassung selbiger, motiviert werden soll. Damit ein Anreiz seine Wirkung entfalten kann, muss ein zugrunde liegendes Bedürfnis existieren, welches der Befriedigung bedarf. \citep[S. 78]{Nowka.2013}

\subsubsection{Zeitlicher Charakter einer Belohnung}
Erhält eine Person eine Belohnung für eine Handlung, wird das Belohnungssystem aktiviert (siehe \ref{sec:Belohnungssystem}). Nach neurowissenschaftlichen Erkenntnissen gibt es dabei jedoch keinen Unterschied, ob die Person intrinsisch oder extrinsisch motiviert ist. \citep[S. 71]{Nowka.2013}
Ebenso führen sowohl eine erwartete wie auch eine unerwartete Belohnung zur selben Reaktion (siehe \ref{sec:Belohnungssystem}). Die Wirkung einer unerwarteten Belohnung ist dabei jedoch deutlich stärker,  bereits die Ankündigung einer Belohnung löst im Belohnungssystem die gleiche Reaktion aus. Bei Erhalt selbiger bleibt das Belohnungssystem jedoch reaktionslos. Das bedeutet, dass das Gehirn nicht zwischen einer unerwarteten und einer angekündigten Belohnung unterscheidet. \citep[S. 71f]{Nowka.2013}
 
Auch der Zeithorizont spielt bei einer Belohnung eine große Rolle. Bei einer Befragung von Probanden, die sich zwischen einer kurzfristigen Belohnung sofort, oder einer großen Belohnung später entscheiden müssen, wählten alle die große Belohnung. Bei einem praktischen Experiment jedoch wählte der überwiegende Teil der Probanden die kurzfristige Belohnung (Marshmallow-Experiment). So bevorzugen die meisten Menschen eine sofortige bzw. zeitnahe Belohnung gegenüber einer späteren, dafür größeren Belohnung. Daraus ergibt sich, dass das Belohnungssystem stärker ist als die Vernunft, die im Entscheidungssystem sitzt. \citep[S. 19]{Seelbach.2011}

\subsubsection{Lerneffekte bei Belohnungen}
Bei mehrfacher Belohnung, insbesondere wenn diese einem regelmäßigen zeitlichen Intervall folgt, besteht ein Lerneffekt. Die Wirkung der Belohnung nimmt ab, da sie als \glqq normal\grqq angesehen wird. Somit muss für die gleiche Wirkung eine höhere Belohnung gewährt werden. \citep[S. 80]{Nowka.2013}
 
Fällt eine zuvor angekündigte Belohnung aus, wird dies vom Gehirn wie ein Verlust behandelt und führt zu Wut und Frustration. Darüber hinaus kommt ein ähnlicher Lerneffekt, wie dem bei häufiger Belohnung, zum Tragen. Dieser sorgt dafür, dass bei einer erneuten Ankündigung einer Belohnung, die Erfahrung aus der zuvor Ausgefallenen, deren Wirkung stark beeinträchtigt. Verstärkt wird die Beeinträchtigung zusätzlich dadurch, dass negative Emotionen wie Wut und Frust, schneller und stärker im Gedächtnis verankert werden. Dies führt zur Priorisierung negativ behafteter Erinnerungen bei der Entscheidungsfindung. Ferner werden negative Erfahrungen meist an andere Personen weiter erzählt. Durch diese Wiederholung der Erinnerung wird deren negative Verankerung zusätzlich gestärkt. \citep[S. 76ff]{Nowka.2013}

\subsubsection{Art der Belohnung}
Bei der Art der Belohnung muss zwischen abstrakter und direkter Belohnung unterschieden werden. Während eine konkrete, greifbare Belohnung direkt als solche wahrgenommen wird, muss eine abstrakte Belohnung zuerst durch einen kognitiven Prozess verarbeitet werden, wodurch sie einen Teil ihrer Wirksamkeit einbüßt. \citep[S. 79f]{Nowka.2013}

\subsubsection{Höhe der Belohnung}
\label{sec:HöheDerBelohnung}
Der Zusammenhang aus Art der Tätigkeit und Höhe der Belohnung hat einen merkbaren Einfluss auf die Leistung einer Person. Bei geistig herausfordernden Tätigkeiten führt eine zu hohe Belohnung zu einer geringeren Leistung, da der Gedanke an die Belohnung ablenkt und sich Anzeichen von Verlustängsten zeigen. Bei primitiven Tätigkeiten hingegen konnte dieser Effekt nicht festgestellt werden. \citep[S. 80f]{Nowka.2013}

\subsubsection{Anreize im sozialen Umfeld}
\label{sec:AnreizeImSozialenUmfeld}
Das Gerechtigkeitsempfinden des Menschen hat bei der Belohnung einer Person innerhalb einer Gruppe großen Einfluss. Generell wird die Belohnung des Einen als Bestrafung des Anderen empfunden. Jeder Mensch strebt nach Gleichbehandlung oder relativer Besserstellung. Das bedeutet, dass ein großer relativer Unterschied der Belohnung innerhalb einer Gruppe als so ungerecht empfunden wird, dass ein genereller Verzicht auf die Belohnung bevorzugt wird. Somit ist die Wirkung der empfundenen Gerechtigkeit größer als die ökonomische Vernunft. Dies beschränkt sich nicht nur auf materielle bzw. monetäre Anreize, sondern erstreckt sich auch auf die soziale Interaktion bzw. Integration innerhalb einer Gruppe. \citep[S. 81ff]{Nowka.2013}
 
Eine mindestens ebenso große Rolle spielt der soziale Status bzw. das zugrunde liegende Anschlussmotiv. Eine positive Änderung des sozialen Status entfaltet im Gehirn eine ähnliche Reaktion wie beispielsweise ein monetärer Anreiz. Insgesamt kommt es jedoch auf Grund des sozialen Drucks generell eher zu einer leichten Minderung der Leistung. \citep[S. 84]{Nowka.2013}

