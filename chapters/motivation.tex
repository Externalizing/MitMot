\section{Einleitung}

Gegenwärtig finden sich in den deutschen Medien vermehrt Beiträge zum Thema Mitarbeitermotivation mit Bezug auf die Herausforderungen in Unternehmen und der Zukunft, bedingt durch den Fachkräftemangel und eine älter werdende Gesellschaft. \citep{Zeit.03.04.2014}

Die Volksweisheit \glqq Geld allein macht nicht glücklich\grqq kennt wahrscheinlich jeder und trotzdem finden sich in den Unternehmen überwiegend Anreizsysteme auf monetärer Basis. Führungskräfte machen im Sinne des Unternehmens meistens alles richtig und doch machen sie nicht das Richtige. Der Faktor Mensch bleibt oft auf der Strecke. \citep[S. 16]{Seelbach.2011}

Viele Unternehmen haben erkannt, dass motivierte Mitarbeiter ein nicht zu unterschätzender Erfolgsfaktor sind. Der Einsatz von monetären Anreizsystemen ist weit verbreitet und eine Maßnahme, die Unternehmen gerne einsetzen um steuernd auf die Motivation ihrer Mitarbeiter einzuwirken. Oft in Vertretung eines variablen Gehaltsanteils, wird dem Mitarbeiter eine hohe Leistung extra vergütet. Genau dieser variable Gehaltsanteil steht bei vielen Mitarbeitern jedoch nicht für  eine Extramotivation, sondern für einen hohen Leistungsdruck. Oft ist das Fixgehalt so niedrig, dass der variable Gehaltsanteil als fester Bestandteil der Entlohnung einkalkuliert werden muss. Damit stellt sich die Frage, welche Instrumente Unternehmen noch zur Motivation ihrer Mitarbeiter außerhalb, bzw. in Kombination zu monetären Anreizsystemen, zur Verfügung stehen. \citep{Nowka.2013}

Ein besonderes Augenmerk wird auf die Erkenntnisse der Neurowissenschaften des  20.Jahrhunderts bis heute gelegt. Hierzu trugen unter anderem die Entwicklung von bildgebenden Messmethoden bei, die es ermöglichen die Aktivitäten im Gehirn direkt darzustellen. Dadurch ist es möglich, nicht nur das äußerliche Verhalten von Probanden zu beschreiben, sondern auch die im Gehirn ablaufenden Prozesse. 

Das Ziel dieser Arbeit ist die Entwicklung eines Konzepts für Unternehmen, die Maßnahmen zur Förderung der Mitarbeitermotivation ergreifen wollen um nachhaltig erfolgreich zu sein. Als Grundlage dienen dabei die klassischen Motivationstheorien und die Erkenntnisse der neurowissenschaftlichen Forschung.

\newpage