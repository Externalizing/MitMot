\section*{Zusammenfassung}
\justifying
In deutschen Medien finden sich vermehrt Beiträge zur Mitarbeitermotivation in Unternehmen. Das zeigt, dass viele Unternehmen motivierte Mitarbeiter als ein nicht zu unterschätzender Erfolgsfaktor erkannt haben. Dabei wird allerdings allzu oft vom \glqq homo oeconomicus\grqq\ ausgegangen, wonach Menschen rein rational handeln. \newline
Mit den Grundlagen der Motivationstheorien und den Erkenntnissen aus der neurowissenschaftlichen Forschung wird die Wirkung von Anreizen, insbesondere auf die Motivation von Menschen, untersucht.
Daraus abgeleitet werden Handlungsanweisungen für Unternehmen gegeben, was diese bei Maßnahmen zur Mitarbeitermotivation beachten sollten. \newline
Die Erkenntnis, dass Anreize sehr individuell und situativ wirken, betont wie wichtig Kenntnisse über die Person sind, die motiviert werden soll. Um dem notwendigen Gerechtigkeitsempfinden des Menschen im Unternehmen als soziales Umfeld nachzukommen, sollte bei den Maßnahmen eines Anreizsystems zur Mitarbeitermotivation auf Transparenz geachtet werden. Als gutes Beispiel kann hier das Cafeteria-Modell genannt werden.
